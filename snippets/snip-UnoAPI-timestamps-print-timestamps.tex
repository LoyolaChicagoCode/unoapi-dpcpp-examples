\begin{cppcode*}{firstnumber=13}
void print_timestamps(const ts_vector & timestamps, const std::string_view filename, const std::string_view device_name) {
    using std::chrono::nanoseconds;
    using std::chrono::duration_cast;

    constexpr auto ROW_HEADER{"TIME,DELTA,UNIT,DEVICE,PHASE\n"};
    constexpr auto ROW_FORMAT{"{},{},{},{},{}\n"};
    constexpr auto TIME_UNIT{"ns"};

    const auto & start = timestamps.front().second;
    auto outfile = filename.empty() ? stdout : std::fopen(filename.data(), "w");
    fmt::print(outfile, ROW_HEADER);
    fmt::print(outfile, ROW_FORMAT, duration_cast<nanoseconds>(start.time_since_epoch()).count(), 0, TIME_UNIT, device_name, timestamps.front().first);
    for (auto t = timestamps.begin() + 1; t != timestamps.end(); t++) {
        const auto dur{duration_cast<nanoseconds>(t->second - (t - 1)->second).count()};
        fmt::print(outfile, ROW_FORMAT, duration_cast<nanoseconds>(t->second.time_since_epoch()).count(), dur, TIME_UNIT, device_name, t->first);
    }
    const auto & stop{timestamps.back().second};
    const auto total{duration_cast<nanoseconds>(stop - start).count()};
    fmt::print(outfile, ROW_FORMAT, duration_cast<nanoseconds>(stop.time_since_epoch()).count(), total, TIME_UNIT, device_name, "TOTAL");
    if (! filename.empty())
        std::fclose(outfile);
}
\end{cppcode*}
